%% deifnições das normas do pdf

\documentclass[12pt, a4paper]{ltxdoc}
\usepackage[T1]{fontenc}		
\usepackage[utf8]{inputenc}	%Codificação UTF-8
	
\usepackage{amsmath}	%Expressoes matematica	
\usepackage{amssymb}
\usepackage{microtype} 			
\usepackage{morefloats}	
\usepackage{mathrsfs}

% Babel e ajustes
\usepackage{booktabs}
\usepackage{titlesec} %Padronizar os titulos
\usepackage{bookmark}
\usepackage{graphicx}
\usepackage{epstopdf}
%\setlength\parindent{20pt}
\usepackage{hyperref}  
\usepackage{subfigure}
\usepackage[titles,subfigure]{tocloft} 
\usepackage{indentfirst}
\usepackage{wrapfig}

\usepackage{cases}
\newtheorem{mydef}{Teorema}
\newtheorem{mylema}{Lema}
\newtheorem{myDef}{Definição}
\newtheorem{exemplo}{Exemplo}
\usepackage{arydshln}
\pagestyle{myheadings}
\markright{}  

\usepackage{color}

\usepackage{setspace}
\usepackage{pslatex} %Fonte times new roman
\usepackage[brazil]{babel}		
\addto\captionsbrazil{
    \renewcommand{\indexname}{\'Indice}
    \renewcommand{\listfigurename}{\hfill \textbf{LISTA DE FIGURAS}\hfill\mbox{} \vspace{1cm}}
    \renewcommand{\listtablename}{Lista de tabelas}
    %% ajusta nomes usados com a macro \autoref
    \renewcommand{\pageautorefname}{p\'agina}
    \renewcommand{\sectionautorefname}{se{\c c}\~ao}
    \renewcommand{\subsectionautorefname}{subse{\c c}\~ao}
    \renewcommand{\paragraphautorefname}{par\'agrafo}
    \renewcommand{\subsubsectionautorefname}{subse{\c c}\~ao}
    \renewcommand{\paragraphautorefname}{subse{\c c}\~ao}
    \renewcommand{\contentsname}{\hfill \textbf{SUMÁRIO} \hfill\mbox{} \vspace{1cm}}
    %\renewcommand{\bibname}{REFERÊNCIA}
    \renewcommand{\refname}{\hfill \textbf{REFER\^ENCIAS} \hfill\mbox{} \vspace{1cm}}
}
\RequirePackage{setspace}      % pacote para controlar o espacamento
\onehalfspace                  % espacamento de 1,5 
\setlength{\parindent}{1.25cm} % recuo do paragrafo
%\numberwithin{equation}{section}     % numerar equação por seção 
\pagenumbering{arabic} % estilo da numera das pg
%Edita a lista de figuras
\newlength{\mylen}
\renewcommand{\cftfigpresnum}{\figurename\enspace}
\renewcommand{\cftfigaftersnum}{\, - \,}
\settowidth{\mylen}{\cftfigpresnum\cftfigaftersnum}
\addtolength{\cftfignumwidth}{\mylen}

\usepackage{color}
\definecolor{thered}{rgb}{0.65,0.04,0.07}
\definecolor{thegreen}{rgb}{0.06,0.44,0.08}
\definecolor{thegrey}{gray}{0.5}
\definecolor{theshade}{rgb}{1,1,0.97}
\definecolor{theframe}{gray}{0.6}
\definecolor{blue}{RGB}{41,5,195}

\IfFileExists{listings.sty}{
  \usepackage{listings}
\usepackage{color} %red, green, blue, yellow, cyan, magenta, black, white
\definecolor{mygreen}{RGB}{28,172,0} % color values Red, Green, Blue
\definecolor{mylilas}{RGB}{170,55,241}
\definecolor{myyellow}{RGB}{255,255,224}
\lstset{language=Matlab,%
    basicstyle=\ttfamily\small,
    breaklines=true,%
    morekeywords={matlab2tikz},
    keywordstyle=\color{blue},%
    morekeywords=[2]{1}, keywordstyle=[2]{\color{blue}},
    identifierstyle=\color{black},%
    stringstyle=\color{mylilas},
    commentstyle=\color{mygreen},%
    showstringspaces=false,%without this there will be a symbol in the places where there is a space
    numbers=left,%
    numberstyle={\tiny \color{black}},% size of the numbers
    numbersep=9pt, % this defines how far the numbers are from the text
    emph=[1]{for,end,break},emphstyle=[1]\color{blue}, %some words to emphasise
    %emph=[2]{word1,word2}, emphstyle=[2]{style}, 
    backgroundcolor=\color{myyellow},
  breakautoindent=true, 
  literate={á}{{\'a}}1 {ã}{{\~a}}1 {é}{{\'e}}1 {è}{{\`{e}}}1 {ê}{{\^{e}}}1 {ë}{{\¨{e}}}1 {É}{{\'{E}}}1 {Ê}{{\^{E}}}1 {û}{{\^{u}}}1 {ú}{{\'{u}}}1 {â}{{\^{a}}}1 {à}{{\`{a}}}1 {á}{{\'{a}}}1 {ã}{{\~{a}}}1 {Á}{{\'{A}}}1 {Â}{{\^{A}}}1 {Ã}{{\~{A}}}1 {ç}{{\c{c}}}1 {Ç}{{\c{C}}}1 {õ}{{\~{o}}}1 {ó}{{\'{o}}}1 {ô}{{\^{o}}}1 {Õ}{{\~{O}}}1 {Ó}{{\'{O}}}1 {Ô}{{\^{O}}}1 {î}{{\^{i}}}1 {Î}{{\^{I}}}1 {í}{{\'{i}}}1 {Í}{{\~{Í}}}1,
}


\let\verbatim\relax
 	\lstnewenvironment{verbatim}[1][]{\lstset{##1}}{}
}

%\usepackage[alf]{abntex2cite}	% citacoes
\usepackage[abnt-emphasize=bf,alf]{abntex2cite}
\setcounter{secnumdepth}{5}%5 seções
\setcounter{tocdepth}{5}

\usepackage{hyperref}
\usepackage{atveryend}

\renewcommand{\cftbeforesecskip}{0.5ex}% espaçamento entre as seções
% poe linhas pontilhadas nas seções do sumario
\renewcommand{\cftsecleader}{\normalfont\bfseries\cftdotfill{\cftsubsecdotsep}} %pontos em negrito
%\renewcommand{\cftsecpagefont}{\normalfont\normalsize\bfseries}
\renewcommand{\cftpartleader}{\normalfont\bfseries\cftdotfill{\cftsubsecdotsep}} %pg em negrito
%\renewcommand{\cftpartpagefont}{\normalfont\normalsize\bfseries}
\renewcommand{\cftsubsubsecleader}{\normalfont\bfseries\cftdotfill{\cftsubsecdotsep}}
\renewcommand{\cftpartleader}{\normalfont\bfseries\cftdotfill{\cftsubsubsecdotsep}}
\renewcommand{\cftsubsubsecpagefont}{\normalfont\normalsize\bfseries}

\titleformat{\paragraph}
{\normalfont\normalsize\bfseries}{\theparagraph}{1em}{}
\titlespacing*{\paragraph}{0pt}{3.25ex plus 1ex minus .2ex}{1.5ex plus .2ex}


\newlength{\ABNTsignwidth}
\setlength{\ABNTsignwidth}{8cm}
\newlength{\ABNTsignthickness}
\setlength{\ABNTsignthickness}{1pt}
\newlength{\ABNTsignskip}
\setlength{\ABNTsignskip}{1cm}
\makeatletter
\newcommand{\assinatura}{\@ifstar{\ABNTsign}{\ABNTcsign}}
\makeatother
% assinatura com estrela
\newcommand{\ABNTsign}[1]{%
  \parbox[t]{\ABNTsignwidth}{\singlespacing\vspace*{\ABNTsignskip}\centering%
  \rule{\ABNTsignwidth}{\ABNTsignthickness}\\%
  \nopagebreak #1\par}%
}



%Sem recuo no sumario
%\renewcommand{\cftchapindent}{0pt}
\usepackage[titles]{tocloft}
%\cftsetindents{chapter}{0cm}{1.5cm}
\cftsetindents{section}{0cm}{0.5cm}
\cftsetindents{subsection}{0cm}{0.8cm}   
\cftsetindents{subsubsection}{0cm}{1cm} 
\cftsetindents{paragraph}{0cm}{1.4cm}
\cftsetindents{figure}{0cm}{2.2cm} %espaçamento na lista de figuras
%\cftsetindents{table}{0em}{3.5em}
\renewcommand{\cftbeforesecskip}{1.0ex}% espaçamento entre as seções no sumario

%editar as secoes 

\renewcommand{\cfttoctitlefont}{\clearpage\thispagestyle{empty}\hfil\bf\MakeUppercase}
\renewcommand{\cftloftitlefont}{\clearpage\thispagestyle{empty}\hfill\bf\MakeUppercase}
\renewcommand{\cftlottitlefont}{\clearpage\thispagestyle{empty}\hfill\bf\MakeUppercase}
\renewcommand{\cftafterloftitle}{\hfill}
\renewcommand{\cftafterlottitle}{\hfill}

\renewcommand{\cftsubsubsecfont}{\bfseries} %subsubsec em negrito

\RequirePackage{pdfpages}
\RequirePackage{ifthen}
%\newcommand{\folhadeaprovacao}{
%\iflogvar
%  \includepdf{Doc1.pdf}
%\else
%  \clearpage\thispagestyle{empty}\mbox{}\vfill\begin{center}\begin{Huge}Folha de aprova\c{c}\~{a}o\end{Huge}\vfill%\end{center}
%\fi
%}

\hypersetup{
		colorlinks=true,
    	linkcolor=black,
    	citecolor=black,
		urlcolor=black
}

\EnableCrossrefs
\CodelineIndex
\RecordChanges

\usepackage[top = 3 cm, bottom = 2 cm, left= 3 cm, right=2 cm]{geometry} %Margens

\titleformat*{\section}{\normalsize}
\titleformat*{\subsection}{\normalsize}
\titleformat*{\paragraph}{\normalsize}

%\numberwithin{equation}{section} %irá prefixar o númeo da seção em todos os números das equações. 