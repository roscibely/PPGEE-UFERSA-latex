\normalsize 
\begin{titlepage}
\onehalfspacing 
 \begin{figure}
 \centering
  	\includegraphics[scale=0.37]{figuras/logo-ufersa.png}
%\end{wrapfigure}
 \end{figure}
  

\setlength\parindent{0pt}
\begin{center}

 UNIVERSIDADE FEDERAL RURAL DO SEMI-ÁRIDO \\
 PRÓ-REITORIA DE PESQUISA E PÓS-GRADUAÇÃO \\
PROGRAMA DE PÓS-GRADUAÇÃO EM ENGENHARIA ELÉTRICA \\ 
MESTRADO EM ENGENHARIA ELÉTRICA
%COMUNICAÇÃO E AUTOMAÇÃO
\end{center}
\vfill

  
\begin{center}
  {{NOME}}\\[4cm]
\end{center}

\begin{center}
   {{{\textbf{TITULO}}}}\\[4cm]
   \end{center}


  \vfill

\vspace{2cm}
\begin{center}

 {MOSSORÓ - RN}
\\
 {ANO}
\end{center}

\end{titlepage}

\onehalfspacing 
\thispagestyle{empty}
\begin{center}
  {{NOME}}\\[4cm]
\end{center}

\begin{center}
   {{{\textbf{TITULO}}}}\\[4cm]
   \end{center}

   \hspace{.45\textwidth} %posiciona a minipage
   \begin{minipage}{.5\textwidth}
   \onehalfspacing 
   Dissertação de mestrado acadêmico apresentada ao Programa de Pós-Graduação em Engenharia Elétrica, como requisito
 à obtenção do título de mestre em
Engenharia Elétrica. Área de Concentração:
Sistema de controle e automação.\\[1cm]
Orientador: Profº. Dr.  - UFERSA
  \end{minipage}
  \vfill

\vspace{2cm}
\begin{center}
{MOSSORÓ - RN}
\\
{2019}
\end{center}
\vspace{0.2cm}
\begin{minipage}{\textwidth}
   \onehalfspacing 
   
\textcolor{thegrey}{O presente trabalho foi realizado com apoio da Coordenação de Aperfeiçoamento de Pessoal de Nível Superior - Brasil (CAPES) - Código de Financiamento 001.}
  \end{minipage}